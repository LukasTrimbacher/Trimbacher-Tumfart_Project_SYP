\documentclass[12pt]{article}
\usepackage{geometry}                % See geometry.pdf to learn the layout options. There are lots.
\geometry{letterpaper}                   % ... or a4paper or a5paper or ... 
\usepackage{graphicx}
\usepackage{amssymb}
\usepackage{amsthm}
\usepackage{epstopdf}
\usepackage[utf8]{inputenc}
\usepackage[usenames,dvipsnames]{color}
\usepackage[table]{xcolor}
\usepackage{hyperref}
\DeclareGraphicsRule{.tif}{png}{.png}{`convert #1 `dirname #1`/`basename #1 .tif`.png}

\theoremstyle{definition}
\newtheorem{example}{Example}

\newenvironment{explanation}{%
   \setlength{\parindent}{0pt}
   \itshape
   \color{blue}
}{}

\newcommand{\projectname}{FinanceM}
\newcommand{\productname}{Finance application for yout Smartphone}
\newcommand{\projectleader}{C. Tumfart, L. Trimbacher}
\newcommand{\documentstatus}{In process}
%\newcommand{\documentstatus}{Submitted}
%\newcommand{\documentstatus}{Released}
\newcommand{\version}{V. 1.0}

\begin{document}
\begin{titlepage}
\begin{flushright}
\includegraphics[scale=.5]{htlleondinglogo.png}\\
\end{flushright}

\vspace{10em}

\begin{center}
{\Huge Project Proposal} \\[3em]
{\LARGE \productname} \\[3em]
\end{center}

\begin{flushleft}
\begin{tabular}{|l|l|}
\hline
Project Name & \projectname \\ \hline
Project Leader & \projectleader \\ \hline
Document state & \documentstatus \\ \hline
Version & \version \\ \hline
\end{tabular}
\end{flushleft}

\end{titlepage}
\section*{Revisions}
\begin{tabular}{|l|l|l|}
\hline
\cellcolor[gray]{0.5}\textcolor{white}{Date} & \cellcolor[gray]{0.5}\textcolor{white}{Author} & \cellcolor[gray]{0.5}\textcolor{white}{Change} \\ \hline
&C. Tumfart/L. Trimabcher&First version \\ \hline
\end{tabular}
\pagebreak

\tableofcontents
\pagebreak

\section{Introduction}
Your FinanceM application should be a helping hand to have an overview about your finances. The user can document his expenditure, bills, income and his money flow. 
The application is going to have an userfriendly interface which makes it more easier to work with it even if you are not into technologie.
After all it should be an application which is easy to handle that helps the user to have a better overview about his in and outgoings.
\pagebreak

\section{Initial Situation}
\begin{explanation}
The initial situation presents the assessment of the actual situation of an organizational unit or the entire organization of an agency or company. Thus a need for action, which may lead to a product or system vision, is recognizable. The vision may be developed into a project idea. The need for action may be initiated by several project or system ideas.
The demonstration of capability gaps (i.e. the difference between the necessary planned capabilities and the actually existing capabilities) in a company or agency may clearly show an urgent need for action in order to increase the efficiency or reduce costs. This need for action is presented as product or system idea, leading frequently to a concrete project proposal. Correspondingly, the determination of the requirement to renew or improve a "technically obsolete" system (so-called "system regeneration") or the recognition of market chances for a new product or system may lead to a project idea. The applicable data must be developed for the project proposal.
Research programs or studies may also be the basis for project ideas; they will be concretized in a project proposal.

The basic question could be summarized in German as follows:
\begin{itemize}
	\item Die Ist-Fähigkeiten der Organisation (was können wir?)
	\item Die Soll-Fähigkeiten der Organisation (was wollen wir können?)
	\item Ein Soll-Ist-Fähigkeitenvergleich (wo liegen die Defizite?)
	\item Ein Fähigkeitsvergleich nach vorgegebenen Bewertungskriterien
\end{itemize}
\end{explanation}

\begin{example}
A doctor in a (primary) school examines the primary school students periodically. (S)he must report the results of this examinations and inform parents about medical defects. Sometimes the doctor has to refer students in problematic health condition to specialists.

The creation of the paperwork is time-consuming since the health data collected during the examinations has to be edited several times (once for the reports, once for the parents information, once for referral, etc. The information retrieval concerning the medical history of students is also time-consuming. The time spent on this work would be better used on direct contact time with students.

The documentation of the examinations can be automated. So the reports and further documents will be generated. The medical history will be persisted automatically.
\end{example}
You are a busy person which is not this much into technologie?
Thats is no problem, with the FinanceM application our team is working on the target to make it as easy as possible to creat an overview for the user about his in and outgoings.

\pagebreak

\section{General Conditions and Constraints}
\begin{explanation}
This subject describes the framework conditions to be observed by all stakeholders when the project idea is implemented into concrete measures for realizing the system. Framework conditions, e.g., budget situation, existing know-how, legal provisions, cooperations, commitment to partners and deadlines, may be turned into specifications for project execution.
Technical framework conditions, e.g., development environments and platforms, IT infrastructure, applicable standards and regulations, or specifications of off-the-shelf products, lead to additional (non-functional) requirements for system development.
\end{explanation}

\begin{example}
The proposed system has to the deal with the following constraints:
\begin{itemize}
\item The information about the medical condition of the pupils is strictly confidential.
\item The GUI of the information system must be intuitive.
\item The application must have a small footprint and a local database.
\item A backup concept is mandatory
\item The application is multi-language capable (english and german)
\end{itemize}
\end{example}

To run our application smoothly and without technical incidents we recommand:
\begin{itemize}
	\item The language is english
	\item An easy usable userinterface
	\item Saving userdata on the device and not at an server for more security
	\item Use of PSD2 to synchronise the application with your bank account
\end{itemize}

\pagebreak

\section{Project Objectives and System Concepts}
\begin{explanation}
In the Subject Project Objectives and System Concepts, the acquirer describes his vision of a new project or system on a high abstraction level. Project objectives and system concepts may concern several aspects, e.g., the introduction of innovations, the definition of objectives (quality, deadline and cost objectives), the operation of the system in its operating environment and the use of new, improved functionalities.
\end{explanation}

\begin{example}
The project objectives can be summarized as follows:
\begin{itemize}
\item The doctor is documenting the examination results while examining the students
\item Input form assists her/him to input information in a structured and easy way
\item Common situations (need for vaccinations, check for need of dental brace, etc.) are a one-click-job for the doctor
\item Info sheet for parents can be printed right after examination
\item Report is a one-click-job at the end of the day
\end{itemize}
\end{example}

The objectives of our project are the following:
\begin{itemize}
\item Its easy for the user to enter the informations in a structured way
\item Compared overview for the user
\item In and outgoings are collected into groups(food, mobility, etc. )
\end{itemize}

\pagebreak
\section{Opportunities and Risks}
\begin{explanation}
The Subject Opportunities and Risks comprises data which are normally prepared in industrial business plans. Frequently, an anonymous market with potential acquirers, which could be interested in the new product or system idea, will be analyzed at first. Therefore, the contents of this subject is characterized by a certain uncertainty or fuzziness. The subjects examines the chances of achieving profit on the market with a specific product or system. In addition to the chances, the risks of failing on the market or sustaining losses with a product or system should be analyzed.
\end{explanation}

\begin{example}
The project has the following opportunities:
\begin{itemize}
\item The doctor is able to increase his time with his patients.
\item The time for bureaucratic work declines.
\item The quality will increase
\end{itemize}

The following risk have to be taken into account.
\begin{itemize}
\item Data transfer of studentsÕ master data from legacy systems is problematic.
\item There is no information about the legacy systems and their data structure.
\item Further there is no information, whether the staff is capable and willing to supply the students master data (names, classes, ...).
\end{itemize}

\end{example}

Our application has the following opportunities:
\begin{itemize}
\item Making it easier for the users to have a overview about their finances
\item Its takes less time than with a pen and paper
\end{itemize}
There are also a few risks:
\begin{itemize}
	\item Its hard to get the live data from the bankaccounts
	\item Storing the data at the local device
\end{itemize}

\pagebreak
\section{Planning}
\begin{explanation}
The planning specifies the organizational and commercial project execution and system development aspects. The project organization, e.g., matrix organization and steering committees, and the responsibilities for the decision-making processes within project will be specified.
The Project Leader will be appointed, his tasks will be defined. Available resources, funds and specialist personnel will be determined. Start and end date for the project will be specified. The planning can be based on the statements developed in the subject Project Objectives and System Concepts, which makes additional statements on feasibility, funding and schedules.

The following parts must be included:
\begin{itemize}
\item List of major project milestones.
\item Assign project lead and other outstanding roles to team members.
\item Give a rough estimate how many resources you need (human resources, licenses, servers, etc.)
\end{itemize}

Answer the following questions when preparing this section:
\begin{itemize}
\item When will the project end?
\item When will the project start?
\item When will be a first prototype available?
\item When does implementation work start?
\item What are the big blocks of work to be done?
\item Is this work doable in the given period of time?
\item Do we need any other stuff to make our work (licenses, servers, É)?
\end{itemize}

List of project milestones

\begin{itemize}	
	\item a rough user interface with some functionalities
	\item simple user login with an server where we save the user data
	\item become access to bank account to get incomes and outcomes
	\item secure user data on the server
	\item make the userinterface cleaner and add more functions to it
\end{itemize}

Project Leaders and an overview of how much resources we need

Project Leaders: Lukas Trimabcher and Christoph Tumfart

Resources we need:

we need 2 people to do the coding and maybe someone to help us with the graphical part of the user interface. Apart from that we need a license so we can get access to the bank accounts of the users.
And of course another server on which we save the user data.

So the project will start on the November 12th this is where we start to implement the userinterface. That would take his time but we think it should be an prototype available after 3-4 weeks.
After we finished the userinterface prototype with and simple user login we have a big chunk to deal with: become access to the bank account to get the incomes and outcomes, the coding is not the big problem in this case, its further the licens from the bank.

When we managed to get a license from the bank and finished the coding, the biggest one is actually done, we just have to make an simple server client communication and need to secure the data wich could be another problem but i think we can manage it.

I think that the biggest problem is the TIME we should get the prototype of the user interface well in time done, but accessing the bank account will be difficult because we do not yet know how and how long it takes to get a license.

\end{explanation}

\end{document}  